
\documentclass[letterpaper,10pt,titlepage]{article}

\usepackage{graphicx}                                        
\usepackage{amssymb}                                         
\usepackage{amsmath}                                         
\usepackage{amsthm}                                          

\usepackage{alltt}                                           
\usepackage{float}
\usepackage{color}
\usepackage{url}

\usepackage{balance}
\usepackage[TABBOTCAP, tight]{subfigure}
\usepackage{enumitem}
\usepackage{pstricks, pst-node}

\usepackage{geometry}
\geometry{textheight=9in, textwidth=6.5in}

%random comment

\newcommand{\cred}[1]{{\color{red}#1}}
\newcommand{\cblue}[1]{{\color{blue}#1}}

\usepackage{hyperref}
\usepackage{geometry}

\def\name{Heather Warman and David Merrick}

%% The following metadata will show up in the PDF properties
\hypersetup{
  colorlinks = true,
  urlcolor = black,
  pdfauthor = {\name},
  pdfkeywords = {cs472 ``computer architecture'' clements  "chapter 3''},
  pdftitle = {CS 472: Homework 3},
  pdfsubject = {CS 472: Homework 3},
  pdfpagemode = UseNone
}

\begin{document}
\hfill \name

\hfill \today

\hfill CS 472 HW 3

\begin{enumerate}
\item[$(3.1)$] Why is the program counter a pointer and not a counter?

The program counter is a pointer because it needs to point to the next instruction to be run. This keeps track of where we are in a program and allows for easy execution of the program. If the program counter was simply a counter, then the program wouldn't know where to go next to implement the instruction.


\item[$(3.2)$] Explain the function of the following registers in a CPU.

a. PC

The PC contains the address of the next instruction to be executed. So it points to the location in memory that holds the next instruction.

b. MAR

The MAR (memory address register) stores the address of the location in main memory that is currently being accessed by a read or write operation.

c. MBR

The MBR (memory buffer register) stores data that has just been read from main memory or data to be immediately written to main memory.

d. IR

The IR (instruction register) stores the instruction most recently read from main memory. This is the instruction currently being executed.


\item[$(3.3)$] For each of the following 6-bit operations, calculate the values of the C,Z,V, and N flags.

\begin{center}
\includegraphics[height=330px]{33}
\end{center}

\item[$(3.10)$] Why does the ARM provide a reverse subtract instruction RSB r0,r1,r2, that implements [r0] = [r2] - [r1] when the normal subtraction instruction SUB r0,r2,r1 will do exactly the same job?

The reason that RSB exists in ARM is that ARM does not have a negation instruction, so you could never just say NEG r1, or essentially just negate a register. With only SUB, it woud be impossible to get the negative value of a register, unless you already knew what the value held in that register was, then you could use SUB r0, r1, \#(negative of r1). In most cases, we don't want to hardcode thsi value, or won't even know this value. With the addition of RSB into the instruction set, we can have \#(some number) - r1, in other words, we can just subtract the value in a register from a number.


\item[$(3.17)$]ARM instructions have a 12-bit literal. Instead of permitting a word in the range 0 to 2\^12 - 1, the ARM uses an 8-bit format for the integer and a 4-bit alignment field that allows the integer to be shifted in steps of 2. What are the advantages and disadvantages of this mechanism in comparison to straight 12-bit integer?

The advantage of the ARM implementation of literals is that is provides range, the disadvantage is that we lose a bit of precision. This implementation is essentially how floating point numbers are stored, where we have a store values in smaller forms with an exponent to multiply it by. In doing this, we lose the precision because our 8-bit literal can only be in the range of 0 to 255 instead of having the full 12-bits to specify a literal, which would result in a range of 0 to 4095. So although we gain the ability to shift bits and extend the range of the 8-bit value, we lose 4 bits of precision.


\item[$(3.18)$]Write one or more ARM instructions that will clear bits 20 to 25 inclusive in register r0. All the other bits of r0 should remain unchanged.

1111 1100 0000 1111 1111 1111 1111 1111 1111

0xFC0FFFFFF

AND r0,r0,0x0xFC0FFFFFF


\item[$(3.19)$]This is a classic problem of assembly language programming. Write a sequence of ARM instructions that swap the contents of registers r0 and r1 without using any additional registers or memory storage (that is, you can't move r1 to a temporary location).

EOR r0,r0,r1

EOR r1,r0,r1

EOR r0,r0,r1


\item[$(3.25)$]What is the binary encoding of the following instructions?

a. STRB r1, [r2]

1110 0101 1100 0010 0001 0000 0000 0000

b. LDR r3, [r4,r5] !

1110 0111 1011 0100 0011 0000 0000 0101 

c. LDR r3, [r4], r5

1110 0110 1001 0100 0011 0000 0000 0101

d. LDR r3, [r4, \#6] !

1110 0111 1100 0001 0010 0000 0000 0011


\item[$(3.39)$]Write an ARM assembly language program that scans a string from a source location pointed at by r0 to a destination pointed at by r1.

	AREA Homework3, CODE, READWRITE

	ENTRY
	
CR		EQU		0x0D	
	
Start

		LDR		r0,Source

		;ADR		r0,Source

		;LDR		r1,Destination

		ADR		r1,Destination

		;STR		r0,[r1]

	
Loop

		LDRB	r2,[r0,r3]		;load r3th byte from r0 into r2

		STRB	r2,[r1,r3]		;store byte from r2 in r1 at r3th position

		CMP 	r2,\#CR			;break if end of string

		BEQ		Complete		

		ADD		r3,r3,\#1		;increment counter

		BAL 	Loop			;loop again


Source

		ALIGN 4

		DCB 	"This is a string",CR

		

Destination

		ALIGN 4

		DCB		""



Complete		

		

	END



\item[$(3.51)$]Write an ARM assembly language program to determine whether a string of characters with an odd length is a palindrome (for example, mom) under the following constraints.

a. The string of ASIC-encoded characters is stored in memory.

b. At the start of the program, register r1 contains the address of the first character in the string, and r2 contains the address of the last character of exit from the program, register r0 contains a 0 is the string is not a palindrome, and 1 if it is.


		AREA Lab2, CODE, READWRITE

		ENTRY
		

CR		EQU 0x0D

	

Start		MOV r0,\#0			;Point the registers to the memory locations

		MOV r1,\#0			;Make sure these are initially zero'd out

		MOV r2,\#0

		MOV r3,\#0			;Use this as loop counter and str length

		MOV r5,\#0			;Holds the current position of char to be compared

		MOV	r6,\#0			;Hold the length of the string

		MOV	r7,\#0			;Use as temp register?

		ADR r4,hello

		

		

Loop			;While not last char, get another byte

		LDRB	        r1,[r4,r3]		;Read n-th byte

		CMP		r1,\#CR		;Compare to the string terminator

		BEQ		Loop\_End		;If == to CR, stop

		ADD		r3,r3,\#1		;If =/= to CR, increment count

		BAL		Loop			;Loop again



Loop\_End	

		LDR	r6,[r3]			;Store length of string in r6 (value of r3, not address,

							;specified with the []

		SUB		r3,r3,\#1

		

		CMP		r6,\#0		;Compare str length to 0

		BEQ		Zero\_Case	;Break if length of string = 0

		CMP		r6,\#1		;Compare str length to 1

		BEQ		One\_Case		;Break if length of string = 1

		B		Div\_Two

		

		

Div\_Two			;Finds half the length of the string, so we know how many times to check bytes

		CMP		r3,\#0

		BEQ		Load\_Half	  	;If r3 == 0, then length of array is 0, break to Zero\_Case

		SUB		r3,r3,\#2		;Else, subtract 2 from value of r3

		ADD		r7,r7,\#1		;Increment counter

		BAL		Div\_Two





Load\_Half	

		LDR		r3,[r7]	        ;Load the value of r7 into r3, which is half the length of the array

							;In other words, we now know how many times to loop through and compare

							;bytes of string

		B		Cmp\_Bytes

		



Cmp\_Bytes

		CMP		r3,r5

		BEQ		One\_Case	        ;If r3 and r5 hold the same value, then we've gone through the entire list, and

							;everything matches, send to One\_Case

		LDRB	        r1,[r4,r5]	        ;Point r1 at the first byte of string

		LDRB	        r2,[r4,r6]	        ;Point r2 at the last byte of the string

		CMP		r1,r2

		BNE		Zero\_Case	;If not equivalent, exit and set r0 to 0

		SUB		r6,r6,\#1		;Subtract one (because we already checked the last bit)

		ADD		r5,r5,\#1		;Add one (because we already checked first bit)

		BAL		Cmp\_Bytes	;Keep looping

		



Zero\_Case	;Set r0 to 0, meaning string was not a palindrome

		MOV r0,\#0

	

One\_Case	        ;Set r0 to 1, meaning string was a palindrome

		MOV r0,\#1





hello		;AREA Data, DATA, READWRITE

		DCB "MOMOM",CR				;Assign labels for each memory location



Done

		END					;Done



\end{enumerate}




\end{document}
   D. Kevin McGrath 
   Last modified: Mon Mar 31 09:26:37 2014